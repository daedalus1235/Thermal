%! TEX root = 0-main.tex
\chapter{Classical Ideal Gas}
The ideal gas will be used to introduce the idea of \emph{entropy}. In an ideal gas, the only interaction between particles is elastic collitions; vdW effects and etc.\  are not considered. The ideal gas approximation is useful for gases at low densities.

In an ideal gas, the particles occupy a negligible volume compared to the volume of the gas; they are essentially point particles. These particles obey Newton's laws of motion, and collisions are elastic and take a negligible amount of time. The particles are allowed to collide with each other and the walls of the container.

For a collection of \(N\) particles in a volume \(V\), each particle is characterized by a position \(\vv{r}_j\) and momentum \(\vv{p}_j\), which each have 3 degrees of freedom.
These positions and momenta form collections \(q=\{\vv{r}_j|j=1,\dots,N\}\) and \(p=\{\vv{p}_j|j=1,\dots,N\}\). Together, \(q\) and \(p\) form the phase space, where points \(\{q,p\}\) indicate a specific microstate.

The kinetic energy of a particle is given \(E_j=\frac{\norm{\vv{p}_j}^2}{2m}\). Thus, the total energy of the system is given:
\begin{equation}
	E=\sum_j^N\frac{\norm{\vv{p}_j}^2}{2m}\label{eq2:energy}
\end{equation}
There are no interactinos between particles, so the potential energy \(U=0\). The particles are also considered to be indistinguishable.

Taking advantage of the fact that particles are considered to be indistinguishable, probaility theory and statistics are used to examine the properties of the classical ideal gas. 
Naturally, assumptions on the distributions of positions and momenta will have to be made; these assumptions will be taken to be as simple as possible, and eveything we don't know is considered to be equally likely.
These assumptions will be validated by computing their consequences and comparing to experimental data.
