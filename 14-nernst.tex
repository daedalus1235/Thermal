%! TEX root = 0-main.tex
\chapter{Nernst Postulate}
Recall that the Nernst Postulate supposes that the entropy of a thermodynamic system goes toa constant as the temperature approaches absolute zero; beginning from quantum mechanics, this postulate arises naturally, but in classical mechanics requires it to be true.
Most interestingly, it shows the impact of quantum mechanics on the values of macroscopic properties.

\section{Classical Ideal Gas Violation}
Recall the entropy of the classical ideal gas. Re-expressing using the equipartition theorem,
\begin{equation}
	S(T,V,N) = k_B N \left[\frac{3}{2}\left(\frac{3}{2}k_B T\right)+\ln\left(\frac{V}{N}\right)+x\right]
\end{equation}
Note that because \(T\) occurs within a natural logarithm, as \(T\to0\), \(S\to-\infty\); in fact, this occurs for all classical systems. As such, quantum mechanics must be used in the study of thermodyamics

\subsection{Planck Extension}
Planck's extension states that the constant that entropy approaches is zero. While this is often true, there are some exceptions; howerver, Nernst's Postulate is \emph{always} true.

\section{Consequences}
The nernst postulate imposes constraints on macroscopic quantities at low temperature; namely, specific heat and thermal expansion.

\subsection{Specific Heat}
Recall the definition of specific heat as
\[c_X(T)=\frac{T}{N}\left(\pder{S}{T}\right)_{X,N}\]
for \(X=V,P\). The change in entropy between two temperatures is given
\[S(T_2)-S(T_1)=\int_{T_1}^{T_2}\d{S} = \int_{T_1}^{T_2}\left(\pder{S}{T}\right)_{X,N}\d{T}=\int_{T_1}^{T_2}\frac{Nc_X(T)}{T}\d{T}\]
Suppose, FSOC that \(\displaystyle\lim_{T\to 0}c_X(T)=c_0\neq 0\). Then, for small temperatures,
\[\Delta S \approx \int_{T_1}^{T_2}\frac{Nc_0}{T}\d{T}=Nc_0\ln(T_2/T_1)\]
Fixing \(T_2\) and sending \(T_1\to 0\), we see that \(\Delta S\to+\infty\), violating the Nernst Postulate, and leading to a contradiction. Thus, we must have \(c_0=0\), and specific heats become zero at absolute zero.

Recall for the Classical Ideal Gas, \(c_V=\frac{3}{2}k_B\) is a constant, independent of temperature; this contradicts the previously derived result, and is another example of how the Classical Ideal Gas violates the Nernst Postulate.

\subsection{Thermal Expansion}
A similar result can be obtained for the coefficient of thermal expansion:
\[\alpha = \frac{1}{V}\left(\pder{V}{T}\right)_{P,N}\]
We can rewrite this using a maxwell relation on the Gibbs Free Energy, so
\[-V\alpha = \left(\pder{S}{P}\right)_{T,N}\]
Because \(S\) is constant at \(T=0\), then this derivative must necessarily be zero at \(T=0\). Thus, we must have \(\alpha\to0\) as \(T\to 0\).
