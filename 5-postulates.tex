%! TEX root = 0-main.tex
\chapter{Postulates and Laws of Thermodynamics}
Statistical mechanics and thermodynamics are different ways to approach thermal physics. 
The same equations can be derived from either theory, but they are not necessarily with the same ease.
The different origins lead to different assumptions between the theories.

Statistical mechanics assumes that the particles exist and obey laws of classical and quantum mechanics. The statistics on these particles give justification for the laws of thermodynamics. Thermodynamics on the other hand, was developed from more experimental evidence, and used postulates to derive behaviours. 

Statistical mechanics deals with the exact state of a system, or an exact point in phase space. In contrast, thermodynamics deals with the macroscopic state, which can be characterised by a handful of variables. This macroscopic state can correspond to a collection of different microscopic states, but it cannot be used to determine the precise microscopic state. 

An equilibrium state is a special kind of a macroscopic state; it does not chnage with time. This equilibrium state can be totally characterized by a small number of variables. For example, the ideal gas law is one way to characterize the equilibrium state for an ideal gas.

A \emph{state function} is a function of a small number of parameters that describes an equilibrium state. One of the most important state functions, which we have already discussed, is \emph{entropy}.

\section{Postulates}
The postulates of thermodynamics that are used in Swendson are:
\begin{enumerate}
	\item There exist equilibrium states that are characterized uniquely by a small number of extensive variables.
	\item There exists a state function (entropy) such that equilibrium states maximise this state function.
	\item The total entropy of two systems is the sum of the individual entropies.
	\item Entropy is a continuous and differentiable 
	\item (Optional) Entropy is an extensive property of a system.
		\[S(\lambda E, \lambda V, \lambda N)= \lambda S(E,V,N)\]
	\item (Optional) Entropy is a monotonically increasing function of energy for equilibrium values of energy.
	\item (Optional) Entropy is always non-negative
\end{enumerate}
Note that postulate \#2 implies the concept of irreversibility, and consequently the arrow of time. 
Additionally, note that postulate \#3 ignores interactions between particles in different subsystem; the hamiltonian includes a coupling term between the boxes.  This is postulate is an approximation, albeit a good one, as interactions are over very short lengths.

The optional postulates are those which are usually true, but there still exist some systems where they do not hold.


\section{Laws}
\begin{enumerate}[start=0]
	\item Two systems in equilibrium with a third are in equilibrium with each other.
	\item Heat is a form of energy transfer, and energy is always conserved.
	\item After the release of a constraint on a system, the entropy can never decrease.
	\item The entropy of a system approaches a constant as temperature approaches zero.
\end{enumerate}

