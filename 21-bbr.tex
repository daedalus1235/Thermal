%! TEX root = 0-main.tex
\chapter{Black Body Radiation}
\section{Resonator Cavity}
Recall Maxwell's equations:
\begin{multicols}{2}
\begin{subequations}\noindent
	\begin{align}
		\del * \vb E & = \frac{\rho}{\varepsilon_0}\\
		\del* \vb B &= 0\\
		\del\times E&=-\pder{\vb B}{t}\\
		\del\times B&=\frac{1}{c^2}\pder{\vb E}{t}+\mu_0\vb J
	\end{align}
\end{subequations}
\end{multicols}
Consider the application of these equations in a vacuum; that is \(\rho = 0\), \(\vb J = 0\). We then have
\[\frac{1}{c^2}\ddot {\vb E} = \del\times\dot{\vb E} = -\del\times\del\times E=-\del(\del*\vb E)+\del^2\vb E = \del^2 \vb E\]
Similarly, we find that
\[\frac{1}{c^2}\ddot{\vb B} = \del^2 \vb B\]
Thus, we see that both \(\vb B\) and \(\vb E\) satisfy the wave equation:
\[\del^2 \left\{\begin{matrix}\vb E \\ \vb B\end{matrix}\right\} - \frac{1}{c^2}\pder{^2}{t^2}\left\{\begin{matrix}\vb E\\\vb B\end{matrix}\right\} = 0\]

We will consider eigenmodes\footnote{We can view the wave equation as an eigenvalue equation for the D'alembert operator \(\Box \vb E = 0\vb E\).} inside a hollow metal cube \([0,L]^3\). Because the walls are an equipotent, we know that \(\vb E\) must be perpendicular to the walls. We guess that the electric field may be written:
\begin{equation}
	\vb E(\vb r, t) = \begin{bmatrix}
		E_{x,0}\cos(k_x x)\sin(k_y y)\sin(k_z z)\\
		E_{y,0}\sin(k_x x)\cos(k_y y)\sin(k_z z)\\
		E_{z,0}\sin(k_x x)\sin(k_y y)\cos(k_z z)
	\end{bmatrix}\sin\omega t
\end{equation}
and indeed, we can verify that this satisfies the wave equation:
\[\del^2 \vb E = -(k_x^2+k_y^2+k_z^2)\vb E\]
\[\pder{^2\vb E}{t^2} = -\omega^2\vb E\]
so long as we have
\[\vb k^2=\frac{\omega^2}{c^2}\]
Note that due to the symmetry of our wave, we need only that
\[\omega = c\abs{\vb k}\]
This is the \emph{dispersion relation} for electromagnetic waves.\footnote{This can be rewritten \(E = \hbar\omega c \abs{\vb k}\). The dispersion relation tells us how wavepackets behave, as we it encodes both the group velocity and phase velocity.}

Further, our guess satisfies the boundary conditions, that \(\vb E\) must be perpendicular to the walls. Indeed, at \(x=0\), the term \(\sin(k_x x)\) kills off all components other than the \(\hat x\), and similarly so for the other axial planes. However, when we consider \(x,y,z = L\), we see that we must have
\[\sin(k_i L)=0\then k_i L = \pi m_i\]
so we have a further constraint that
\[k_i = \frac{\pi m_i}{L}\qquad m_i\in \Z\]
combining this with the wave equation, we obtain
\[\omega = \frac{\pi c}{L}\sqrt{m_x^2+m_y^2+m_z^2}\]
Once again, due to the symmetry of the wave, we need only consider \(m_i \in \N\setminus\{0\}\). Thus,
\begin{equation}
	\omega = \frac{\pi c m}{L}
\end{equation}
where \(\vb m\in \N^3\setminus\{0\}\) is called the \emph{mode}. However, we need one more parameter to specify the state---the \emph{polarization} of the wave, \(\sigma\in\{1,2\}\). It is interesting to note that the allowed states in the box are discretized. 

\subsection{Density of States}
Consider the density of states in mode space:
\[D(\vb m) = 2\sum_{\vb n} \delta(\vb m-\vb n)\]
where the factor of 2 is inculded due to the polarization. We can extend this to a continuous variable, so
\[D(\vb m) = 2\]
We also want the DoS in frequency space; this can be obtained by
\begin{align*}
	D(\omega) &= \int\d[3]{\vb m} D(\vb m) \delta\left(\omega-\frac{\pi c m}{2}\right)\\
		  &= \frac{4\pi}{8}\int\d{m} m^2D(m)\delta\left(\omega-\frac{\pi c m}{2}\right)\\
		  &= \frac{\pi}{2}*2 \left(\frac{L}{\pi c}\right)^3\int\d{\tilde m}\tilde m^2 \delta \left(\omega - \tilde m\right)\\
		  &=\left(\frac{L}{\pi c}\right)^3\pi\omega^2
\end{align*}
The factor of \(\pi\omega^2\) can be viewed as a multiplicity constant due to the portion of a spherical shell in the first octant: \(\frac{1}{8}*2*4\pi\omega^2\), keeping in mind the polarization factor 2.

\section{Quantization}
We notice that every eigenmode of the EM field is essentially a harmonic oscillator of frequency \(\omega\); following Planck, we consider each of these oscillators to be as a collection of quantum mechanical oscillators. The individual hamiltonian then becomes
\[\hat H_{\vb m, \sigma} = \hbar\omega_{\vb m}\left(\hat n_{\vb m, \sigma}+\frac{1}{2}\right)-\frac{1}{2}\hbar\omega_{\vb m}\]
where the second term is to neglect the ground-state energy in analogy to the classical expectation. And so, the total hamiltonian of the electromagnetic field in the box can be given
\begin{equation}
	\hat H = \sum_{\vb k, \sigma} \left[\hbar c k \left(\hat n_{\vb k, \sigma}+\frac{1}{2}\right)-\frac{1}{2}\hbar ck\right]
\end{equation}
and so the EM field is equivalent to two quantum harmonic oscillators at every valid momentum space point. Note that our calculation has two different quantizations. First, is that the classical boundary conditions and their corresponding eigenmodes enforce a discretized, but not quantum, momentum space. However, the quantization of the energy levels of the harmonic oscillators is the imposition of a quantum mechanical effect. The energy expectation for a single oscillator, is of course,
\[\vect{\varepsilon_{\vb k, \sigma}} = \frac{\hbar\omega}{e^{\beta\hbar\omega}-1}\]
Thus, the spectrally resolved energy density is
\[u_\omega = \frac{1}{V}D_\omega(\omega)\vect{\varepsilon_{\omega}}\]
where we divide by volume to obtain a density, and the remaining terms are the energy of a state times the density of that state. 
Substituting and simplifying, we obtain
\begin{equation}
u_\omega = \frac{\hbar\omega^3}{\pi^2 c^3}*\frac{1}{e^{\beta\hbar\omega}-1}
\end{equation}

\subsection{Energy Flux}
If we poke a hole into the box and allow radiation to escape, we obtain an energy flux through the hole is
\[j_\omega = \frac{1}{4}c u_\omega\]
where of course, \(c u_\omega\sim v\rho = j\). The prefactor of \(1/4\) arises due to the average projection over a given direction of a half sphere (particles going left will not exit to the right). Let the hole be pointing in the \(+\hat z\) direction. As photons exit the hole, they point in any direction. The flux along this axis must then be proportional to 
\[\frac{\int_0^{\pi/2}\d\theta\sin\theta\int_0^{2\pi}\d\phi\cos\theta }{\int_0^{\pi/2} \d\theta\sin\theta\int_0^{2\pi}\d\phi}= \frac{1}{4}\]
Thus we obtain 
\begin{equation}
	j_\omega = \frac{\hbar}{4\pi^2c^2}*\frac{\omega^3}{e^{\beta\hbar\omega}+1}
\end{equation}
which is more commonly known as the planck law. At low \(\omega\), this grows \(\sim \omega^2\), which agrees with the classical density of states, but at high \(\omega\), the quantum effects dominate, so
it falls off with \(\sim\omega^2e^{-\beta\hbar\omega}\) which allows the function to be integrable and resolve the UV catastrophe.


As shown in the HW, we can rewrite the density for wavelength space as
\begin{equation}
	j_\lambda = \frac{2\pi hc^2}{\lambda^5}\frac{1}{e^{\beta hc/\lambda}+1}
\end{equation}

Fix \(x=\beta\hbar\omega\) and \(y = \frac{\beta \hbar c}{\lambda}\). Then, we can write the distributions as
\[j_x = A_x \frac{x^3}{e^x-1}\qquad\qquad j_\lambda=A_y \frac{y^5}{e^y+1}\]
Finding the maximum, we obtain \(x_{\max}=3+W_0\left(-\frac{3}{e^3}\right)\approx 2.82144\), where \(W_0\) is the Lambert \(W\) function, or the inverse\footnote{The non-monotonicity of \(xe^x\) means that the inverse is not single-valued. The upper real branch is given \(W_0\) while the lower real branch is \(W_1\). This is, of course, a complex valued function.} of \(f(x) = xe^x\). The maximum flux can be found 
\[j_{\omega_{\max}}=\frac{\hbar}{(2\pi c)^2}\frac{\left(x_{\max} \frac{k_BT}{\hbar}\right)^3}{e^{x_{\max}}-1}\propto T^3\]
The relation that \(\hbar\omega_{\max}\propto T\) is known as Wien's displacement law. Combining these two proportionalities, we see that \(\text{Area}\sim\omega_{\max}j_{\omega_{\max}}\propto T^4\) which is the Stefan-Boltzmann law. We can derive the Stefan-Bolzmann law in another way:
\begin{align*}
	U &=\int_0^\infty\d{\omega}D_\omega(\omega)\vect{\varepsilon_\omega}\\
	  &=\pi\left(\frac{L}{\pi c}\right)^3\int_0^\infty\d{\omega}\frac{\hbar\omega^3}{e^{\beta\hbar\omega}-1}\\
	  &=\pi\left(\frac{L}{\pi c}\right)^3\int_0^\infty \frac{\d{x}}{\beta\hbar}\frac{\hbar(x/\beta\hbar)^3}{e^x-1}\\
	  &=\pi\left(\frac{L}{\pi c}\right)^3\left(\frac{k_BT}{\hbar}\right)^4\hbar\int_0^\infty\d{x}\frac{x^3}{e^x-1}\\
	  \intertext{Even without integrating, we see that \(U\propto T^4\) and \(U\propto V\). To get the multiplicative factor, we need to evaluate the integral. It turns out to have a nice form:}
	  &=\pi\left(\frac{L}{\pi c}\right)^3\left(\frac{k_BT}{\hbar}\right)^4\hbar\frac{\pi^4}{15}\\
	  &=\frac{\pi^2}{15\hbar^3c^3}V(k_BT)^4
\end{align*}
Thus,
\begin{equation}
	\mathcal J_u = \frac{1}{4}c\frac{U}{V}=\sigma T^4\qquad\qquad \sigma = \frac{\pi^2k_B^4}{60\hbar^3c^3}\end{equation}
where \(\sigma\) is the \emph{Stefan-Boltzmann constant}.

\newpage
\begin{aside}[Boltzmann's Derivation]
When we integrate the frequency-resolved flux density, we obtain the Stefan-Boltzmann law, which states that \(u\sim T^4\). Stefan discovered this law empirically, but Boltzmann managed to derive this law purely thermodynamically, long before quantum mechanics. He made three key assumptions which allowed him to do so. First, he assumed that electromagnetic radiation is extensive---double the volume, double the contained radiation. Second, he obtained from Maxwell's laws that the equation of state could be given \(P = \frac{1}{3}u\) (which is derived from, for example, the maxwell stress-energy tensor). Finally, he assumed that the chemical potential of a photon is equally zero, as photons leave with no cost.\footnote{This last assumption we will see holds true because the Planck distribution is a Bose-Einstein distribution with chemical potential zero. In fact, all photons have chemical potential zero.} 

We can then write the energy of the system as
\[U(T,V,\mu) = U(T,v) = Vu(T)\]
The fact that \(\mu\) is equally zero means that we can remove the functional dependence. Further, the extensivity in \(V\) allows us to write the energy as linear in \(V\).

Further, in extensive systems, the Euler relation holds:
\[U = TS-PV+\mu N\]
The final term is again, equally zero. From Maxwell's laws, the pressure of EM radiation is given as a third the energy density, so \(PV=\frac{1}{3}U\). Then, we can write
\[S = \frac{1}{T}\left(U+PV\right)=\frac{4}{3}\frac{1}{T}U = \frac{4}{3}\frac{Vu(T)}{T}\]
Finally, we use the maxwell relation
\[\left(\pder{P}{T}\right)_{V}=\left(\pder{S}{V}\right)_T\]
Substituting each side of the maxwell relation, we see that
\[\frac{1}{3}u'(T) = \frac{4}{3}\frac{u(T)}{T}\then u'(T) = 4\frac{u(T)}{T}\]
From this differential equation, we can easily see that
\[u\propto T^4\]
\end{aside}
